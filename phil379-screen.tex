% phil379-screen.tex

% driver file phil379-screen.tex to produce the Logic II textbook on
% with same type block as in printed version, but but with on-screen
% features (color, links, etc).

\documentclass[screen]{../../sty/open-logic-book}

\definecolor{OLPcolor}{RGB}{248,154,14}
\definecolor{OLPdkcolor}{RGB}{104,65,6}
\definecolor{OLPltcolor}{RGB}{248,203,137}

\usepackage[nonumberlist,toc,style=index]{glossaries}
\makeglossaries

\def\oljobname{phil379}

% \olpath has to point to the location of the OLP main
% directory/folder. We're compiling from subdirectory
% courses/phil379/, so the main directory is two
% levels up.

\newcommand{\olpath}{../../}

% load all the Open Logic definitions. This will also load the
% local definitions in open-logic-sample-config.sty

\input{\olpath/sty/open-logic.sty}

% Metadata for phil379

\setOLPbooktitle{Sets, Logic, Computation}
\setOLPbooksubtitle{An Open Introduction to Metalogic}
\setOLPauthor{Richard Zach}
\setOLPbookversion{Fall 2019\textalpha}{F19\textalpha}
\setOLPsourcelink{https://github.com/rzach/phil379}
\setOLPauthorlink{http://richardzach.org/}

% we want all the problems deferred to the very end

\input{\olpath/sty/open-logic-defer.sty}

% load glossary entries

\loadglsentries{include/glossary}

% end preamble

\begin{document}
\raggedbottom

\OLPscreencover{\hfill\includegraphics[width=.29\textwidth]{\olpath/assets/portraits/cantor-circle.pdf}\hfill{}

\hfill{}
\includegraphics[width=.29\textwidth]{\olpath/assets/portraits/goedel-circle.pdf}
\hfil\includegraphics[width=.29\textwidth]{\olpath/assets/portraits/turing-circle.pdf}\hfil{}
\includegraphics[width=.29\textwidth]{\olpath/assets/portraits/skolem-circle.pdf}
\hfill{}

\hfill{}
\includegraphics[width=.29\textwidth]{\olpath/assets/portraits/noether-circle.pdf}\hfill{}
}

% Now load the actual text

% phil379.tex
%
% driver file phil379.tex to produce text on letter-size paper
% with standard layout and margins

% We use the memoir class for maximal flexibility of layout, but any
% class will do

\documentclass[letterpaper]{memoir}

% \olpath has to point to the location of the OLP main
% directory/folder.  We're compiling from subdirectory courses/sample,
% so the main directory is two levels up.
\newcommand{\olpath}{../../}

% load all the Open Logic definitions. This will also load the
% local definitions in open-logic-sample-config.sty
\input{\olpath/sty/open-logic.sty}

% we want all the problems deferred to the end
\input{\olpath/sty/open-logic-defer.sty}

% let's set the whole thing in Palatino, with Helvetica for
% sans-serif, and spread the lines a bit to make the text more
% readable

\usepackage{mathpazo}
\usepackage[scaled=0.95]{helvet}
\linespread{1.05}

\begin{document}

% First we make a titlepage

\begin{titlingpage}
\begin{raggedleft}
\fontsize{52pt}{2em}\selectfont\bfseries\sffamily
Logic II
\vskip 4ex
\normalfont\Huge\textbf{Edited by \href{http://richardzach.org/}{Richard Zach}}\

\end{raggedleft}

\vfill

% oluselicense generates a license mark that a) licenses the result
% under a CC-BY licence and b) acknowledges the original source (the
% OLP).  Acknowledgment of the source is a requirement under the
% conditions of the CC-BY license used by the OLP, but you are not
% required to license the product itself under CC-BY.

\oluselicense
% Title of this version of the OLT with link to source
{\href{https://github.com/rzach/phil379}{\textit{Logic II}}}
% Author of this version
{\href{http://richardzach.org/}{Richard Zach}}
\end{titlingpage}

\frontmatter
\pagestyle{ruled}

\tableofcontents*

\mainmatter

\olimport*[sets-functions-relations]{sets-functions-relations}

\olimport*[first-order-logic]{first-order-logic}

\olimport*[turing-machines]{turing-machines}

\stopproblems

% now typeset all the problems as an appendix. If you want problems at
% the end of each chapter, delete this part and put
% \problemsperchapter in the preamble

\backmatter

\chapter{Problems}

\printproblems

\end{document}



\end{document}

