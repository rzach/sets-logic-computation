% phil379-screen.tex

% driver file phil379-screen.tex to produce the Logic II textbook on
% with same type block as in printed version, but but with on-screen
% features (color, links, etc).

% We use the memoir class for maximal flexibility of layout, but any
% class will do

\documentclass[11pt,openany]{memoir}

\RequirePackage{amsthm}
\RequirePackage{xcolor}
\RequirePackage{mdframed}
\RequirePackage[full]{textcomp}
\RequirePackage{gitinfo2}

\usepackage[nonumberlist,toc,style=index]{glossaries}
\makeglossaries

% let's set the whole thing in Baskervald X, with Universalis ADF
% Standard for sans-serif, and spread the lines a bit to make the text
% more readable

\usepackage[sfdefault]{universalis}
\usepackage[osf]{Baskervaldx} % oldstyle figures

\usepackage[bigdelims,baskervaldx]{newtxmath} 
\usepackage[cal=boondoxo]{mathalfa}

% Make sure we have a copyright symbol

\def\copyright{\textcircled{C}}

\def\oljobname{phil379}

% next line needed for Adobe Acrobat to open the PDF. may have
% something to do with transparency in the PNG graphics files?

\pdfminorversion=4

% load textpos for the cover

\usepackage[absolute,overlay]{textpos}

% set stock & paper size with narrow margins

\setstocksize{8in}{5in}

\settrimmedsize{\stockheight}{\stockwidth}{*}
\settrims{0pt}{0pt}

% let's calculate the line length for 65 characters in \normalfont

\setlxvchars

% set the size of the type block to golden ratio calculated width

\settypeblocksize{*}{1.05\lxvchars}{1.62}

% set spine and and edge margin

\setlrmargins{*}{*}{1}
\setulmargins{.6in}{*}{*}
\setheaderspaces{*}{*}{1}

\checkandfixthelayout

% Chapter style

\makeatletter

\definecolor{leadbeater}{RGB}{248,154,14}
\definecolor{dkleadbeater}{RGB}{104,65,6}
\definecolor{ltleadbeater}{RGB}{248,203,137}
\newlength{\barlength}
\makechapterstyle{leadbeater}{%
  \setlength{\afterchapskip}{40pt}
  \setlength{\beforechapskip}{50pt}
    \setlength{\midchapskip}{10pt}
  \renewcommand*{\afterchapternum}{\par\nobreak\vskip 0pt}
  \renewcommand*{\chapnamefont}{\fontsize{14pt}{0pt}\selectfont\sffamily\bfseries}
  \renewcommand*{\chapnumfont}{\fontsize{14pt}{0pt}\selectfont\sffamily\bfseries}
  \renewcommand*{\chaptitlefont}{\normalfont\fontsize{48pt}{48pt}\selectfont\bfseries\itshape\color{leadbeater}}
  \renewcommand*{\printchaptername}{%
    \chapnamefont\MakeUppercase{\@chapapp}}
  \renewcommand*{\printchaptertitle}[1]{%
    \chaptitlefont ##1\\[-\baselineskip]%
    \hspace*{-20pt}%
    \smash{\color{leadbeater}\rule{0pt}{300pt}}}
}

\renewcommand*{\partnamefont}{\fontsize{24pt}{0pt}\selectfont\sffamily\bfseries}
\renewcommand*{\partnumfont}{\fontsize{24pt}{0pt}\selectfont\sffamily\bfseries}
\renewcommand*{\parttitlefont}{\normalfont\fontsize{54pt}{54pt}\selectfont\bfseries\itshape\color{leadbeater}}
\renewcommand*{\printpartname}{%
  \partnamefont PART}

\makeatother

\chapterstyle{leadbeater}

\copypagestyle{leadbeater}{headings}

\makeoddhead{leadbeater}{\small\sffamily\color{leadbeater}\rightmark}{}
            {\color{leadbeater}\sffamily\bfseries\thepage}

\makeevenhead{leadbeater}{\small\sffamily\color{leadbeater}\leftmark}{}
            {\color{leadbeater}\sffamily\bfseries\thepage}

\usepackage[font={small,it}]{caption}


% \olpath has to point to the location of the OLP main
% directory/folder. We're compiling from subdirectory
% courses/phil379/, so the main directory is two
% levels up.

\newcommand{\olpath}{../../}

% load all the Open Logic definitions. This will also load the
% local definitions in open-logic-sample-config.sty

\input{\olpath/sty/open-logic.sty}

\hypersetup{allcolors=dkleadbeater,
  pdftitle={Sets, Logic, Computation},
  pdfauthor = {Richard Zach, Open Logic Project}}

\RequirePackage[numbered]{bookmark}

% we want all the problems deferred to the very end

\input{\olpath/sty/open-logic-defer.sty}

% load glossary entries

\loadglsentries{include/glossary}

% end preamble

\begin{document}

% First we make a titlepage

\thispagestyle{empty}

  \pagecolor{leadbeater}
  \begin{textblock*}{4.5in}(.25in,.75in)%
    \begin{raggedright}
      \fontsize{26pt}{30pt}\selectfont\bfseries\sffamily%
      Sets, Logic, Computation\\
      \normalfont\fontsize{18pt}{0pt}\selectfont\bfseries\itshape%
      \rule{4.5in}{5pt}\\[5pt]
      An Open Logic Text


    \vskip1.5cm

    \includegraphics[scale=.7]{illustrations/Cover}

    \vskip 1.5cm
    \noindent
    \includegraphics[width=1cm]{assets/cc.pdf}
    \includegraphics[width=1cm]{assets/by.pdf}
    \includegraphics[width=1cm]{assets/remix.pdf}
    \normalfont\fontsize{16pt}{0pt}\selectfont\bfseries\sffamily%
    \hfill Fall 2017 \textit{bis}
    \end{raggedright}
\end{textblock*}
\

\clearpage
\setcounter{page}{1}
\nopagecolor

% Now load the actual text

% phil379.tex
%
% driver file phil379.tex to produce text on letter-size paper
% with standard layout and margins

% We use the memoir class for maximal flexibility of layout, but any
% class will do

\documentclass[letterpaper]{memoir}

% \olpath has to point to the location of the OLP main
% directory/folder.  We're compiling from subdirectory courses/sample,
% so the main directory is two levels up.
\newcommand{\olpath}{../../}

% load all the Open Logic definitions. This will also load the
% local definitions in open-logic-sample-config.sty
\input{\olpath/sty/open-logic.sty}

% we want all the problems deferred to the end
\input{\olpath/sty/open-logic-defer.sty}

% let's set the whole thing in Palatino, with Helvetica for
% sans-serif, and spread the lines a bit to make the text more
% readable

\usepackage{mathpazo}
\usepackage[scaled=0.95]{helvet}
\linespread{1.05}

\begin{document}

% First we make a titlepage

\begin{titlingpage}
\begin{raggedleft}
\fontsize{52pt}{2em}\selectfont\bfseries\sffamily
Logic II
\vskip 4ex
\normalfont\Huge\textbf{Edited by \href{http://richardzach.org/}{Richard Zach}}\

\end{raggedleft}

\vfill

% oluselicense generates a license mark that a) licenses the result
% under a CC-BY licence and b) acknowledges the original source (the
% OLP).  Acknowledgment of the source is a requirement under the
% conditions of the CC-BY license used by the OLP, but you are not
% required to license the product itself under CC-BY.

\oluselicense
% Title of this version of the OLT with link to source
{\href{https://github.com/rzach/phil379}{\textit{Logic II}}}
% Author of this version
{\href{http://richardzach.org/}{Richard Zach}}
\end{titlingpage}

\frontmatter
\pagestyle{ruled}

\tableofcontents*

\mainmatter

\olimport*[sets-functions-relations]{sets-functions-relations}

\olimport*[first-order-logic]{first-order-logic}

\olimport*[turing-machines]{turing-machines}

\stopproblems

% now typeset all the problems as an appendix. If you want problems at
% the end of each chapter, delete this part and put
% \problemsperchapter in the preamble

\backmatter

\chapter{Problems}

\printproblems

\end{document}



\end{document}

