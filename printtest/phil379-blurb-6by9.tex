% phil379-blurb-6by9.tex
%
% driver file phil379.tex to produce text on 6by9
% stock for blurb.com

% We use the memoir class for maximal flexibility of layout, but any
% class will do

\documentclass[10pt]{memoir}

% \olpath has to point to the location of the OLP main
% directory/folder.  We're compiling from subdirectory
% courses/sample/phil379/printtest, so the main directory is three
% levels up.
\newcommand{\olpath}{../../../}

% load all the Open Logic definitions. This will also load the
% local definitions in open-logic-sample-config.sty
\input{\olpath/sty/open-logic.sty}

% all links plain black for printing

\hypersetup{hidelinks}

% we want all the problems deferred to the end
\input{\olpath/sty/open-logic-defer.sty}

% let's set the whole thing in Palatino, with Helvetica for
% sans-serif, and spread the lines a bit to make the text more
% readable

\usepackage{mathpazo}
\usepackage[scaled=0.95]{helvet}
\linespread{1.05}

% set stock & paper size to Blurb's 6"x9"
\setstocksize{9.25in}{6.125in}
\settrimmedsize{9in}{6in}{*}
\settrims{.125in}{.125in}

% let's calculate the line length for 65 characters in \normalfont
\setlxvchars

% set the size of the type block to calculated width in golden ratio

\settypeblocksize{*}{\lxvchars}{1.618}

% set spine and and edge maring in golden ratio

\setlrmargins{*}{*}{1.618}

\setulmargins{60pt}{*}{*}

\setheaderspaces{*}{*}{1.618}

\checkandfixthelayout

\begin{document}

% First we make a titlepage
\newbox\adjust

\begin{titlingpage}
\begin{raggedleft}
  \fontsize{48pt}{7em}\selectfont\bfseries\sffamily
  \setbox\adjust\hbox{\phantom{,}}
Sets,\\
Logic,\\
Computation\usebox\adjust\\
\vskip 4ex
\normalfont\Huge\textbf{Edited by\usebox\adjust\\ \href{http://richardzach.org/}{Richard Zach}\usebox\adjust}

\end{raggedleft}

\vfill

% oluselicense generates a license mark that a) licenses the result
% under a CC-BY licence and b) acknowledges the original source (the
% OLP).  Acknowledgment of the source is a requirement under the
% conditions of the CC-BY license used by the OLP, but you are not
% required to license the product itself under CC-BY.

\oluselicense
% Title of this version of the OLT with link to source
{\href{https://github.com/rzach/phil379}{\textit{Sets, Logic, Computation}}}
% Author of this version
{\href{http://richardzach.org/}{Richard Zach}}
\end{titlingpage}

\frontmatter
\pagestyle{ruled}

\tableofcontents*

\mainmatter

\olimport*[sets-functions-relations]{sets-functions-relations}

\olimport*[first-order-logic]{first-order-logic}

\olimport*[turing-machines]{turing-machines}

\stopproblems

% now typeset all the problems as an appendix. If you want problems at
% the end of each chapter, delete this part and put
% \problemsperchapter in the preamble

\backmatter

\chapter{Problems}

\printproblems

\end{document}

