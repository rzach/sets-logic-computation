% slc-screen.tex

% driver file slc-screen.tex to produce the Sets, Logic, Computation
% textbook on with same type block as in printed version, but but with
% on-screen features (color, links, etc).

\documentclass[screen]{../../sty/open-logic-book}

\definecolor{OLPcolor}{RGB}{248,154,14}
\definecolor{OLPdkcolor}{RGB}{104,65,6}
\definecolor{OLPltcolor}{RGB}{248,203,137}

\usepackage[nonumberlist,toc,style=index]{glossaries}
\makeglossaries

\def\oljobname{slc}

% \olpath has to point to the location of the OLP main
% directory/folder. We're compiling from subdirectory
% courses/sets-logic-computation/, so the main directory is two
% levels up.

\newcommand{\olpath}{../..}

% load all the Open Logic definitions. This will also load the
% local definitions in open-logic-sample-config.sty

\input{\olpath/sty/open-logic.sty}

% Metadata for slc

\setOLPbooktitle[Sets, Logic\rlap{,}\\ Computation]{Sets, Logic, Computation}
\setOLPbooksubtitle{An Open Introduction to Metalogic}
\setOLPauthor{Richard Zach}
\setOLPbookversion{Fall 2025}{F25}
\setOLPsourcelink{https://slc.openlogicproject.org/}
\setOLPauthorlink{https://richardzach.org/}

% we want all the problems deferred to the very end

\input{\olpath/sty/open-logic-defer.sty}

% load glossary entries

\loadglsentries{include/glossary}

% end preamble

\begin{document}
\raggedbottom

\OLPscreencover{\hfill\includegraphics[width=.29\textwidth]{\olpath/assets/portraits/cantor-circle.pdf}\hfill{}

\hfill{}
\includegraphics[width=.29\textwidth]{\olpath/assets/portraits/goedel-circle.pdf}
\hfil\includegraphics[width=.29\textwidth]{\olpath/assets/portraits/turing-circle.pdf}\hfil{}
\includegraphics[width=.29\textwidth]{\olpath/assets/portraits/skolem-circle.pdf}
\hfill{}

\hfill{}
\includegraphics[width=.29\textwidth]{\olpath/assets/portraits/noether-circle.pdf}\hfill{}
}

% Now load the actual text

% slc.tex
%
% driver file slc.tex to produce text

\preto\OLEndChapterHook{\IfFileExists{include/summary-\thechapter}
        {\section*{Summary}\addcontentsline{toc}{section}{Summary}
        \let\emph\textbf\input{include/summary-\thechapter}\let\emph\textit}{}}

\problemsperchapter

\frontmatter

\OLPfrontmatter

\cleartoverso
\thispagestyle{empty}
\ \vfill
\noindent\includegraphics[width=\textwidth]{illustrations/SkolemChapter}
\vfill

\chapter{Preface}\markboth{Preface}{Preface}

This book is an introduction to metalogic, aimed especially at
students of computer science and philosophy. ``Metalogic'' is
so-called because it is the discipline that studies logic itself.
Logic proper is concerned with canons of valid inference, and its
symbolic or formal version presents these canons using formal
languages, such as those of propositional and first-order logic.
Meta-logic investigates the properties of these languages, and of the
canons of correct inference that use them. It studies topics such as
how to give precise meaning to the expressions of these formal
languages, how to justify the canons of valid inference, what the
properties of various !!{derivation} systems are, including their
computational properties. These questions are important and
interesting in their own right, because the languages and
!!{derivation} systems investigated are applied in many different
areas---in mathematics, philosophy, computer science, and linguistics,
especially---but they also serve as examples of how to study formal
systems in general. The logical languages we study here are not the
only ones people are interested in. For instance, linguists and
philosophers are interested in languages that are much more
complicated than those of propositional and first-order logic, and
computer scientists are interested in other \emph{kinds} of languages
altogether, such as programming languages. And the methods we
discuss---how to give semantics for formal languages, how to prove
results about formal languages, how to investigate the properties of
formal languages---are applicable in those cases as well.

Like any discipline, metalogic both has a set of results or facts,
and a store of methods and techniques, and this text covers both.
Many students won't need to know all of the results we discuss
outside of this course, but they will need and use the methods we use
to establish them. The L\"owenheim-Skolem theorem, say, does not
often make an appearance in computer science, but the methods we use
to prove it do. On the other hand, many of the results we discuss do
have relevance for certain debates, say, in the philosophy of science
and in metaphysics. Philosophy students may not need to be able to
prove these results outside this course, but they do need to
understand what the results are---and you really only
\emph{understand} these results if you have thought through the
definitions and proofs needed to establish them. These are, in part,
the reasons for why the results and the methods covered in this text
are recommended study---in some cases even required---for students of
computer science and philosophy.

The material is divided into three parts. \Olref[sfr][][]{part}
concerns itself with the theory of sets. Logic and metalogic is
historically connected very closely to what's called the ``foundations
of mathematics.''  Mathematical foundations deal with how ultimately
mathematical objects such as integers, rational, and real numbers,
functions, spaces, etc., should be understood. Set theory provides one
answer (there are others), and so set theory and logic have long been
studied side-by-side. Sets, relations, and functions are also
ubiquitous in any sort of formal investigation, not just in
mathematics but also in computer science and in some of the more
technical corners of philosophy. Certainly for the purposes of
formulating and proving results about the semantics and proof theory
of logic and the foundation of computability it is essential to have a
terminology in which to do this. For instance, we will talk about sets
of expressions, relations of consequence and provability,
interpretations of predicate symbols (which turn out to be relations),
computable functions, and various relations between and constructions
using them. It will be good to have shorthand symbols for these, and
think through the general properties of sets, relations, and
functions. If you are not used to thinking mathematically and to
formulating mathematical proofs, then think of the first part on set
theory as a training ground: all the basic definitions will be given,
and we'll give increasingly complicated proofs using them. Note that
understanding these proofs---and being able to find and formulate them
yourself---is perhaps more important than understanding the results,
especially in the first part. If mathematical thinking is new to you,
it is important that you think through the examples and problems.

In the first part we will establish one important result, however.
This result---Cantor's theorem---relies on one of the most striking
examples of conceptual analysis to be found anywhere in the sciences,
namely, Cantor's analysis of infinity. Infinity has puzzled
mathematicians and philosophers alike for centuries. Until Cantor,
no-one knew how to properly think about it. Many people even
considered it a mistake to think about it at all, and believed that
the notion of an infinite collection itself was incoherent. Cantor
made infinity into a subject we can coherently work with, and
developed an entire theory of infinite collections---and infinite
numbers with which we can measure the sizes of infinite collections.
He showed that there are different levels of infinity. This theory of
``transfinite'' numbers is beautiful and intricate, and we won't get
very far into it; but we will be able to show that there are different
levels of infinity, specifically, that there are ``countable'' and
``uncountable'' levels of infinity. This result has important
applications, but it is also really the kind of result that any
self-respecting mathematician, computer scientist, and philosopher
should know.

In \olref[fol][][]{part}, we turn to first-order logic. We will define
the language of first-order logic and its semantics, i.e., what
first-order structures are and when a sentence of first-order logic is
true in a structure. This will enable us to do two important things:
(1)~We can define, with mathematical precision, when a sentence is a
logical consequence of another. (2)~We can also consider how the
relations that make up a first-order structure are
described---characterized---by the sentences that are true in them.
This in particular leads us to a discussion of the axiomatic method,
in which sentences of first-order languages are used to characterize
certain kinds of structures. Proof theory will occupy us next, and we
will consider the original version of the sequent calculus and natural
deduction as defined in the 1930s by Gerhard Gentzen. (Your instructor
may choose to cover only one, then any reference to
``!!{derivation}s'' and ``!!{derivability}'' will mean whatever system
they chose.) The semantic notion of consequence and the syntactic
notion of !!{derivability} give us two completely different ways to
make precise the idea that a sentence may follow from some others. The
soundness and completeness theorems link these two characterization.
In particular, we will prove G\"odel's completeness theorem, which
states that whenever a sentence is a semantic consequence of some
others, then it is also !!{derivable} from them. An equivalent
formulation is: if a collection of sentences is consistent---in the
sense that nothing contradictory can be proved from them---then there
is a structure that makes all of them true.

The second formulation of the completeness theorem is perhaps the more
surprising. Around the time G\"odel proved this result (in 1929), the
German mathematician David Hilbert famously held the view that
consistency (i.e., freedom from contradiction) is all that mathematical
existence requires. In other words, whenever a mathematician can
coherently describe a structure or class of structures, then they
should be entitled to believe in the existence of such structures.
At the time, many found this idea preposterous: just because you can
describe a structure without contradicting yourself, it surely does
not follow that such a structure actually exists. But that is exactly
what G\"odel's completeness theorem says. In addition to this
paradoxical---and certainly philosophically intriguing---aspect, the
completeness theorem also has two important applications which allow
us to prove further results about the existence of structures which
make given sentences true. These are the compactness and the
L\"owenheim-Skolem theorems.

In \olref[tur][][]{part}, we connect logic with computability. Again,
there is a historical connection: David Hilbert had posed as a
fundamental problem of logic to find a mechanical method which would
decide, of a given sentence of logic, whether it has a proof. Such a
method exists, of course, for propositional logic: one just has to
check all truth tables, and since there are only finitely many of
them, the method eventually yields a correct answer. Such a
straightforward method is not possible for first-order logic, since
the number of possible structures is infinite (and structures
themselves may be infinite). Logicians were working to find a more
ingenious methods for years. Alonzo Church and Alan Turing eventually
established that there is no such method. In order to do this, it was
necessary to first provide a precise definition of what a mechanical
method is in general. If a decision procedure had been proposed,
presumably it would have been recognized as an effective method. To
prove that no effective method exists, you have to define ``effective
method'' first and give an impossibility proof on the basis of that
definition. This is what Turing did: he proposed the idea of a Turing
machine\footnote{Turing of course did not call it that himself.} as a
mathematical model of what a mechanical procedure can, in principle,
do. This is another example of a \emph{conceptual analysis} of an
informal concept using mathematical machinery; and it is perhaps of
the same order of importance for computer science as Cantor's analysis
of infinity is for mathematics. Our last major undertaking will be the
proof of two impossibility theorems: we will show that the so-called
``halting problem'' cannot be solved by Turing machines, and finally
that Hilbert's ``decision problem'' (for logic) also cannot.

This text is mathematical, in the sense that we discuss mathematical
definitions and prove our results mathematically. But it is not
mathematical in the sense that you need extensive mathematical
background knowledge. Nothing in this text requires knowledge of
algebra, trigonometry, or calculus. We have made a special effort to
also not require any familiarity with the way mathematics works: in
fact, part of the point is to \emph{develop} the kinds of reasoning
and proof skills required to understand and prove our results. The
organization of the text follows mathematical convention, for one
reason: these conventions have been developed because clarity and
precision are especially important, and so, e.g., it is critical to
know when something is asserted as the conclusion of an argument, is
offered as a reason for something else, or is intended to introduce
new vocabulary. So we follow mathematical convention and label
passages as ``definitions'' if they are used to introduce new
terminology or symbols; and as ``theorems,'' ``propositions,''
``lemmas,'' or ``corollaries'' when we record a result or finding.
Other than these conventions, we will use the methods of logical proof
that may already be familiar from a first logic course, and we will
also make extensive use of the method of \emph{induction} to prove
results. Two chapters of the appendix are devoted to these proof
methods.

\paragraph{Notes for instructors} The material in this book is
suitable for a semester-long second course  in formal logic. I cover
it in 12 weeks in Logic~II taught at the University of Calgary,
although I don't cover everything in as much detail as there is in
this book. For instance, I typically only talk about natural
deduction, and leave out detailed proofs of completeness for identity.
Students have taken Logic~I, typically taught from
\emph{\href{https://forallx.openlogicproject.org}{forall x: Calgary}},
which uses the same natural deduction rules, except in Fitch format.

The most recent version of this book is available in PDF at
\href{https://slc.openlogicproject.org}{slc.openlogicproject.org}, but
changes frequently.  The CC BY license gives you the right to download
and distribute the book yourself. In order to ensure that all your
students have the same version of the book throughout the term you're
using it, you should do so: upload the PDF you decide to use to your
LMS rather than merely give your students the link. You are also free
to have the PDFs printed by your bookstore, but some bookstores will
be able to purchase and stock the softcover books available on Amazon.

The syntax, semantics, and proof systems for first-order logic are
supported by Graham Leach-Krouses's free, online logic teaching
software application \emph{Carnap}
(\href{https://carnap.io}{carnap.io}). This allows for submission and
automated marking of exercises such as natural deduction and sequent
calculus !!{derivation}s, giving structures for simple theories, and
symbolization exercises.  There is also a Turing machine simulator at
\href{https://turing.openlogicproject.org}{turing.openlogicproject.org}
that can be used to illustrate the material in \olref[tur][][]{part}.
The examples there are available pre-loaded in the simulator.

\cleartoverso 


\thispagestyle{empty}
\ \vfill
\noindent\includegraphics[width=\textwidth]{illustrations/CantorChapter}
\vfill

\mainmatter

\olimport*[sets-functions-relations]{sets-functions-relations}

\cleartoverso
\ifodd\value{page}\stepcounter{page}\fi
\thispagestyle{empty}
\ \vfill
\noindent\includegraphics[width=\textwidth]{illustrations/GodelChapter}
\vfill

\olimport*[first-order-logic]{first-order-logic}

\cleartoverso
\ifodd\value{page}\stepcounter{page}\fi
\thispagestyle{empty}
\ \vfill
\noindent\includegraphics[width=\textwidth]{illustrations/TuringChapter}
\vfill

\olimport*[turing-machines]{turing-machines}

% appendices

\cleartoverso
\ifodd\value{page}\stepcounter{page}\fi
\thispagestyle{empty}
\ \vfill
\noindent\includegraphics[width=\textwidth]{illustrations/NoetherChapter}
\vfill

\appendix

\olimport*[methods/proofs]{proofs}

\olimport*[methods/induction]{induction}

\stopproblems
\def\ifproblems#1{}

\def\figurename{Fig.}
\setlength{\olphotowidth}{.45\textwidth}

\chapter{Biographies}

\olimport*[history/biographies]{georg-cantor}

\olimport*[history/biographies]{alonzo-church}

\olimport*[history/biographies]{gerhard-gentzen}

\olimport*[history/biographies]{kurt-goedel}

\olimport*[history/biographies]{emmy-noether}

\olimport*[history/biographies]{bertrand-russell}

\olimport*[history/biographies]{alfred-tarski}

\olimport*[history/biographies]{alan-turing}

\olimport*[history/biographies]{ernst-zermelo}

\olimport*[reference/greek-alphabet]{greek-alphabet}

%\backmatter

\glsaddall
\printglossaries

\clearpage
\photocredits

\bibliographystyle{\olpath/bib/natbib-oup}
\bibliography{\olpath/bib/open-logic.bib}

\olimport*{\olpath/content/open-logic-about}


\end{document}